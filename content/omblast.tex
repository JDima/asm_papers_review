% !TEX root = ../review.tex
% \begin{frame}
% \frametitle{OMBlast\nocite{omblast}: Общие сведения}

% \end{frame}

\begin{frame}
\frametitle{OMBlast\nocite{omblast}: Алгоритм}

Этапы выравнивания:
\begin{itemize}
  \item Поиск стартовых мест (сидов) для начала выравнивания
  \item Расширение сидов
  \item Объединение пересекающих выравниваний - построение частичных выравниваний
  \item Построение итогового выравнивания
\end{itemize}

\end{frame}

\begin{frame}
\frametitle{OMBlast: Поиск стартовых сидов - индексация}
  Фрагмент $q$ на карте совпадает с фрагментом $r$ на референсе:
  \begin{equation*}
    r(1 - T_s) - T_m \le q \le r(1 + T_s) + T_m
  \end{equation*}
  $T_s$ - параметр, ошибка масштабирования \\
  $T_m$ - параметр, ошибка измерений \\
  \begin{figure}
    \centering
    \includegraphics[width = 0.9\textwidth]{omblast/index_A}
  \end{figure}
\end{frame}

\begin{frame}
\frametitle{OMBlast: Поиск стартовых сидов - бины}
  \begin{figure}
    \centering
    \includegraphics[width = 0.9\textwidth]{omblast/index_B}
  \end{figure}
\end{frame}

\begin{frame}
\frametitle{OMBlast: Расширение сидов (1)}
  \begin{figure}
    \centering
    \includegraphics[width = 0.9\textwidth]{omblast/ext_seeds_1}
  \end{figure}
  \begin{figure}
    \centering
    \includegraphics[width = 0.9\textwidth, height = 0.4\textheight]{omblast/ext_seeds_2}
  \end{figure}
\end{frame}

\begin{frame}
\frametitle{OMBlast: Расширение сидов (2)}
  \begin{figure}
    \centering
    \includegraphics[width = 0.9\textwidth, height = 0.9\textheight]{omblast/dp_matrix}
  \end{figure}
\end{frame}

\begin{frame}
\frametitle{OMBlast: Объединение выравниваний (1)}
  Строится взвешенный ациклический граф:
  \begin{itemize}
    \item Вершины - выравненные разрезы
    \item Рёбра - между двумя парами последовательно (на одной карте) выравненных разрезов
    \item Веса - $t_m \, u_m - t_{es} \, u_{es} - t_{ms} \, u_{ms}$ \\
    $u_{m}$ - количество совпадений\\
    $u_{es}$ - количество лишних разрезов\\
    $u_{ms}$ - количество пропущенных разрезов
  \end{itemize}
  \begin{figure}
    \centering
    \includegraphics[width = 0.9\textwidth]{omblast/alg_mrg_0}
  \end{figure}
\end{frame}

\begin{frame}
\frametitle{OMBlast: Объединение выравниваний (2)}
  \begin{figure}
    \centering
    \includegraphics[width = 0.9\textwidth]{omblast/alg_mrg_2}
  \end{figure}
  \begin{figure}
    \centering
    \includegraphics[width = 0.9\textwidth]{omblast/alg_mrg_3}
  \end{figure}

\end{frame}

\begin{frame}
\frametitle{OMBlast: Объединение выравниваний (3)}
С помощью динамического программирования определяется путь в графе с наибольшим весом
  \begin{figure}
    \centering
    \includegraphics[width = 0.9\textwidth]{omblast/alg_mrg_1}
  \end{figure}

\end{frame}


\begin{frame}
\frametitle{OMBlast: Построение итогового выравнивания(1)}
\begin{figure}
  \centering
  \includegraphics[width = 0.9\textwidth]{omblast/graph_problem}
\end{figure}

\end{frame}

\begin{frame}
\frametitle{OMBlast: Построение итогового выравнивания(2)}
Имеются два частичных выравнивания:
\begin{figure}
  \centering
  \includegraphics[width = 0.9\textwidth]{omblast/trimming}
\end{figure}
После получаем итоговое выравнивание, для которого можно посчитать значимость
\end{frame}

\begin{frame}
\frametitle{OMBlast: Значимость выравнивания}
Значение выравнивания:
\begin{gather*}
  score = \sum\limits_{i=1}^q p_i - \sum\limits_{i=1}^{q - 1} r_{i,i+1} \\
  p_i = (t_m \, u_m - t_{es} \, u_{es} - t_{ms} \, u_{ms}) (1 - |1 - s|)
\end{gather*}
$p_i$ - частичное выравнивание \\
$t_m, t_{es}, t_{ms}$ - параметры баллов за совпадений, лишний и пропущенный разрез \\
$u_m, u_{es}, u_{ms}$ - количество совпадений, лишних и пропущенных \\
$s$ - масштаб \\
Значимость выравнивания:
\begin{gather*}
  u_i = \frac{score_i - score_{min}}{\sum\limits_{j} (score_j - score _{min} + 1)}
\end{gather*}
\end{frame}

\begin{frame}
\frametitle{OMBlast: Результаты - входные данные}
  \begin{figure}
    \centering
    \includegraphics[width = 0.9\textwidth]{omblast/data}
  \end{figure}
  \begin{figure}
    \centering
    \includegraphics[width = 0.9\textwidth]{omblast/error_levels}
  \end{figure}
\end{frame}

\begin{frame}
\frametitle{OMBlast: Результаты - время работы}
  \begin{figure}
    \centering
    \includegraphics[width = 0.9\textwidth]{omblast/time}
  \end{figure}

\end{frame}

\begin{frame}
\frametitle{OMBlast: Результаты - точность и полнота }
  \begin{figure}
    \centering
    \includegraphics[width = 0.9\textwidth, height = 0.9\textheight]{omblast/error}
  \end{figure}

\end{frame}

\begin{frame}
\frametitle{OMBlast: Вывод }
Плюсы:
\begin{itemize}
  \item Определение SV
\end{itemize}
Минусы:
\begin{itemize}
  \item Предложенные схемы хешинга будут занимать много места
\end{itemize}
\end{frame}

% \begin{frame}
% \frametitle{OMBlast: Результаты - наличие SV}
%   \begin{figure}
%     \centering
%     \includegraphics[width = 0.9\textwidth, height = 0.9\textheight]{omblast/sv}
%   \end{figure}
% \end{frame}
