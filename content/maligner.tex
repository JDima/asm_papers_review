% !TEX root = ../review.tex
\begin{frame}
\frametitle{MAligner: Общие сведения}
Два подхода:
\begin{itemize}
  \item На основе алгоритма Смита-Ватермана
  \begin{enumerate}
    \item Построение множества выравниваний на референсе
    \item Отклонение выравниваний с помощью M-Score
  \end{enumerate}
  \item На основе индексации
\end{itemize}
\end{frame}

\begin{frame}
\frametitle{MAligner: Алгоритм динамического программирования}

Пусть имеются два выравненных участка c n и m пропущенными фрагментами длины r и m на референсе и карте соотвественно.
Тогда выравнивание имеет следующее значение:

\begin{gather*}
Score(q, r, m, n) = S(q, r) + C_q \,m + C_r \, n \\
S(q, r) = \bigg(\frac{q - r}{\sigma(r)}\bigg)^2 \\
\sigma(r) = \max(\alpha \, r, \sigma_{min})
\end{gather*}
$C_q$ - штраф за пропущенные фрагменты на карте \\
$C_r$ - штраф за пропущенные фрагменты на референсе \\
$\sigma_{min}$ - для фрагментов малой длины, ошибка больше \\
$\alpha$ - доля референса, которая будет использовать как стандартное отклонение
\end{frame}

\begin{frame}
\frametitle{MAligner: M-Score - значимость выравнивания}

Предложена оценка M-Score для определения значимости выравнивания:

\begin{gather*}
  m_{\mathcal{A}} = \underset{A \in \mathcal{A}}{median}\{Score(A)\} \\
  MAD_{\mathcal{A}} = \underset{A \in \mathcal{A}}{median}\{ | Score(A) - m_{\mathcal{A}}|\} \\
  M-Score_{\mathcal{A}}(A) = \frac{Score(A) - m_{\mathcal{A}}}{MAD_{\mathcal{A}}}
\end{gather*}

$Score(A)$ - значение выравнивания $A$

$\mathcal{A}$ - 100 лучших выравниваний по Score(A)
\end{frame}


\begin{frame}
\frametitle{MAligner: Алгоритм на основе индексов}

\end{frame}
