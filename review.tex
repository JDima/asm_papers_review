\documentclass{beamer}
%
% Choose how your presentation looks.
%
% For more themes, color themes and font themes, see:
% http://deic.uab.es/~iblanes/beamer_gallery/index_by_theme.html
%
\mode<presentation>
{
  \usetheme{Warsaw}      % or try Darmstadt, Madrid, Warsaw, ...
  \usecolortheme{default}
  \usepackage{beamerthemesplit}% or try albatross, beaver, crane, ...
  \usefonttheme{default}  % or try serif, structurebold, ...
  \setbeamertemplate{navigation symbols}{}
  \setbeamertemplate{caption}[numbered]
  \setbeamertemplate{headline}{}
  \AtBeginSection[]{
    \begin{frame}
    \vfill
    \centering
    \begin{beamercolorbox}[sep=8pt,center,shadow=true,rounded=true]{title}
      \usebeamerfont{title}\insertsectionhead\par%
    \end{beamercolorbox}
    \vfill
    \end{frame}
  }
  \AtBeginSubsection[]{
    \begin{frame}
    \vfill
    \centering
    \begin{beamercolorbox}[sep=8pt,center,shadow=true,rounded=true]{title}
      \usebeamerfont{title}\insertsubsectionhead\par%
    \end{beamercolorbox}
    \vfill
    \end{frame}
  }
}
\usepackage[english,russian]{babel}
\usepackage[utf8]{inputenc}
\usepackage{amsmath}
\usepackage{dutchcal} %math symbols
\usepackage{hyperref}
\usepackage[backend=bibtex]{biblatex}
\bibliography{bibliography.bib}

% Removes icon in bibliography
\setbeamertemplate{bibliography item}{}

\hypersetup{unicode=true}

\title[Обзор алайнеров]{Обзор алайнеров}
\author{Дмитрий Яковлев}
\institute{EPAM Systems}
\date{\today}

\begin{document}
\graphicspath{{./img/}}

\begin{frame}
  \titlepage
\end{frame}

\begin{frame}
\frametitle{План}
\tableofcontents
\end{frame}

\section{Введение}
% !TEX root = ../review.tex
\begin{frame}
\frametitle{Введение}
\textbf{Цель} \\
Найти наилучший для ассемблирования алайнер для датасета от BioNano.\\
\textbf{Алайнер:}
\begin{figure}
  \centering
  \includegraphics[width = 0.9\textwidth]{intro/aligner}
\end{figure}
\textbf{Ассемблер:}
\begin{figure}
  \centering
  \includegraphics[width = 0.9\textwidth]{intro/assembler}
\end{figure}


\end{frame}


\section{Модель ошибки на данных BioNano}
% !TEX root = ../review.tex
\begin{frame}
\frametitle{Модель ошибок: общие сведения}
\begin{itemize}
  \item Было рассмотрено 3 датасета карт от BioNano
  \item С помощью RefAligner был построен референс
  \item Далее был проведён анализ ошибок
\end{itemize}
\end{frame}

\begin{frame}
\frametitle{Модель ошибок: ошибка в длине фрагмента}


\begin{figure}
\centering
\begin{minipage}{.5\textwidth}
  Валуев:
  \begin{gather*}
  e_k = \frac{o_k - r_k}{\sqrt{r_k}} \sim N(0, \sigma) \\
  o_k \sim N(r_k, \sigma^2 \, r_k)
  \end{gather*}
  \centering
  \includegraphics[width=.9\linewidth]{sigma_error_norm}
\end{minipage}%
\begin{minipage}{.5\textwidth}
  Новый подход:
  \begin{gather*}
  s_k = \frac{o_k}{r_k} \\
  s_k \sim Laplace(\mu, \beta)
  \end{gather*}
  \centering
  \includegraphics[width=.9\linewidth]{sigma_error_laplace}
\end{minipage}
\end{figure}
\end{frame}

\begin{frame}
\frametitle{Модель ошибок: пропущенные разрезы}

\end{frame}

\begin{frame}
\frametitle{Модель ошибок: лишние разрезы}

\end{frame}


\section{Алайнеры}

\subsection{TWIN}
% !TEX root = ../review.tex

\begin{frame}
\frametitle{TWIN\nocite{twin}: Алгоритм}
Алгоритм разработан в предположении отсутствия пропущенных и лишних разрезов.
Идея:
\begin{itemize}
  \item Построение FM-индекса на референсе
  \item Выравнивание карты на референсе будем искать как подстроку в строке
  \item В статье предлагается алгоритм неточного поиска подстроки в строке
\end{itemize}


\end{frame}


\subsection{OPTIMA}
% !TEX root = ../review.tex
\begin{frame}
\frametitle{OPTIMA: Общие сведения}


\end{frame}

\begin{frame}
\frametitle{OPTIMA: Алгоритм}

Этапы выравнивания:
\begin{itemize}
  \item Поиск стартовых мест (сидов) для начала выравнивания
  \item Парное выравнивание карты с референсом
  \item Определение значимых выравниваний
  \item Объединение пересекающихся выравниваний
\end{itemize}

\end{frame}


\begin{frame}
\frametitle{OPTIMA: Определение значимости выравнивания}
Пусть $a$ - выравнивание из множества выравниваний $\mathcal{A}$
\begin{gather*}
Z-score(a \in \mathcal{A}, f) = \frac{f_{a} - Mean(f_{\mathcal{A}})}{SD(f_{\mathcal{A}})}
\end{gather*}
где $f$ - характеристика выравнивания.\\
 Тогда статистическая значимость выравнивания:
\begin{align*}
  \vartheta (a \in \mathcal{A}) = Z-score( & -Z-score(a, \#matches) \\
  & + Z-score(a, \#cuterrors) \\
  & + Z-score(a, WHT(\chi^2, \#matches))) \\
\end{align*}
\[\text{где } WHT(\chi^2, \#matches) = \frac{\sqrt[3]{\frac{\chi^2}{\#matches}} - \big(1 - \frac{1}{9} \frac{2}{\#matches}\big)}{\sqrt{\frac{1}{9} \frac{2}{\#matches}}}\]
\end{frame}


\subsection{MAligner}
% !TEX root = ../review.tex
\begin{frame}
\frametitle{MAligner: Общие сведения}
Два подхода:
\begin{itemize}
  \item На основе алгоритма Смита-Ватермана (за основу взята статья Нагараяна)
  \item На основе индексации
\end{itemize}
\end{frame}

\begin{frame}
\frametitle{MAligner: Алгоритм динамического программирования}

\end{frame}

\begin{frame}
\frametitle{MAligner: Алгоритм на основе индексов}

\end{frame}


\subsection{OMBlast}
% !TEX root = ../review.tex
\begin{frame}
\frametitle{OMBlast}
Этапы выравнивания:
\begin{center}

  \begin{itemize}
    \item Поиск стартовых мест (сидов) для начала выравнивания
    \item Расширение сидов
    \item Объединение пересекающих выравниваний
    \item Построение итогового выравнивания
  \end{itemize}

\end{center}
\end{frame}

\begin{frame}
\frametitle{OMBlast: Поиск стартовых сидов}

\end{frame}

\begin{frame}
\frametitle{OMBlast: Расширение сидов}

\end{frame}

\begin{frame}
\frametitle{OMBlast: Объединение пересекающих выравниваний}


\end{frame}

\begin{frame}
\frametitle{OMBlast: Построение итогового выравнивания}


\end{frame}


\section{Ссылки}
\begin{frame}
\frametitle{Ссылки: исходники}
В открытом доступе \nocite{*}:
\begin{itemize}
  \item \href{http://www.cs.colostate.edu/twin/download.html}{TWIN}
  \item \href{https://github.com/verznet/OPTIMA}{OPTIMA}
  \item \href{https://github.com/LeeMendelowitz/maligner}{MAligner}
  \item \href{https://github.com/aldenleung/OMBlast}{OMBlast}
\end{itemize}
\end{frame}

\begin{frame}[t,allowframebreaks]
\frametitle{Ссылки: статьи}
\printbibliography
\end{frame}

\begin{frame}

\begin{center}
\Huge Спасибо за внимание!
\end{center}

\end{frame}

\end{document}
